\documentclass[12pt]{article}
\usepackage[margin=1in]{geometry}

% PACKAGES
\usepackage{amsmath} % For extended formatting
\usepackage{amssymb} % For math symbols
\usepackage{amsthm} % For proof environment
\usepackage{array} % For tables
\usepackage{enumerate} % For lists
\usepackage{extramarks} % For headers and footers
\usepackage{fancyhdr} % For custom headers
\usepackage{graphicx} % For inserting images
\usepackage{multicol} % For multiple columns
\usepackage{verbatim} % For displaying code
\usepackage{tkz-euclide}
\usepackage{pgfplots}

% SET UP HEADER AND FOOTER
\pagestyle{fancy}
\lhead{\MyCourse} % Top left header
\chead{\MyTopicTitle} % Top center header
\rhead{\MyAssignment} % Top right header
\lfoot{\MyCampus} % Bottom left footer
\cfoot{} % Bottom center footer
\rfoot{\MySemester} % Bottom right footer
\renewcommand\headrulewidth{0.4pt} % Size of the header rule
\renewcommand\footrulewidth{0.4pt} % Size of the footer rule
\newcommand{\MyCourse}{Math 241}
\newcommand{\MyTopicTitle}{Integrals}
\newcommand{\MyAssignment}{Name: \qquad \qquad \qquad}
\newcommand{\MySemester}{FALL 2019}
\newcommand{\MyCampus}{University of Hawaii at Manoa}
\begin{document}
\section{Definition of Integral}
An integral of a function is the anti-derivative of it. 
\begin{equation}
    \int f(x) dx = F(x) + C
\end{equation}
\textbf{Caution} If the integral is indefinite \textbf{(No specific bounds)} then write down the letter C. 
\subsection{Properties of the Integral}
\begin{equation}
    \int k f(x) dx = k \int f(x) dx
\end{equation}
where k is any constant.
\begin{equation}
    \int -f(x) dx = - \int f(x) dx 
\end{equation}
\begin{equation}
    \int f(x) \pm g(x) dx= \int f(x) dx \pm \int g(x) dx  
\end{equation}
\begin{equation}
    \int a dx = ax + C
\end{equation} where a is a constant 

\begin{equation}
    \int x^{n} = \frac{x^{n+1}}{n+1}
\end{equation}
\subsection{Properties of Definite Integrals}
\textbf{Definite Integrals} means an integral with a certain bounds. 
Written as $\int_{a}^{b} f(x) dx$ where f(x) represents a certain function and b and a represents the boundary areas. \newline 
\begin{equation}
    \int_{a}^{b} f(x) dx = - \int_{b}^{a} f(x) dx 
\end{equation}
\begin{equation}
      \int_{a}^{a} f(x) dx = 0
\end{equation}
\begin{equation}
    \int_{a}^{b}f(x)dx \pm \int_{b}^{c}f(x)dx = \int_{a}^{c} f(x)dx
\end{equation}
where a, b, and c represents a boundary point. 
\section{Integration by Substitution}

\section{Trigonometry Integrals}
\begin{itemize}
    \item $\int sin(x) dx = -cos(x)+C$
    \item $\int cos(x) dx = sin(x)+C$
    \item $\int sec^{2}(x)dx = tan(x)+C$
    \item $\int csc^{2}(x)dx = cot(x)+C$
    \item $\int sec(x)tan(x) dx = sec(x)+C$
    \item $\int csc(x)cot(x) dx = -csc(x) + C $
\end{itemize}

\end{document}
