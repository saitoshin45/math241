\documentclass{article}

\title{Precalculus Review}
\author{University of Hawaii at Manoa}
\date{Fall 2019 Math 241}
\begin{document}
\maketitle 
\text{This covers the basic elements in Precalculus which should be aid to help you in Math 241}
\begin{center}
    \textbf{Why do I need to care about Math 241?}
\end{center}
Depending on your major(s) Math 241 is going to be "important". For example, if you are required to take Physics 170, you are going to need Math 242 as a cor-requisite. In order to take Math 242, you need to take Math 241 and get a C or above. \newline 
Here are some of the classes that require Math 241 as a prereq.
\begin{itemize}
    \item Physics 170 
    \item ICS 141 Discrete Mathematics I 
\end{itemize}
\newpage
\section{Algebra}
\textbf{Exponents}
\begin{itemize} 
    \item Multiplying Exponents \newline
    \text{ If the exponent value have the same base but different exponent you can add the numerical value in the exponent}\newline
    \textbf{Example} \newline
    \begin{center}
    $
    x^{2}x^{4} =x^{6}
    $
    \end{center}
    \newline
    \begin{center} 
    $
    4x^{4}5x^{2}=20x^{6}
    $
    \end{center}
    \item Dividing Exponents \newline
    If the bases are the same but with different exponent, you can subtract the two exponents. \newline 
    \textbf{Example} \newline
    \begin{center}
        $
        \frac{x^{4}}{x^{2}}=x^{2}
        $
    \end{center}
    
\end{itemize}

\textbf{Fractions}
\begin{itemize}
    \item adding/subtracting \newline
    \textbf{Example}\newline 
    \begin{center}
    $
    \frac{4}{8}+\frac{3}{6}-\frac{1}{24}=
    $    
    \end{center}
    \textbf{Process} \newline 
    Check if the denominator (the bottom part of the fraction) are the same and 
    if not change the denominator so that they "common denominators" \newline
    \textbf{Warning}\newline 
    When you change the denominator do not forget to change the numerator \newline 
    Then you do the math operations 
    \begin{center}
    $
    \frac{12}{24}+\frac{12}{24}-\frac{1}{24}=\frac{23}{24}
    $
    \end{center}
    
    \item multiplying fractions \newline 
    Unlike addition, there is no need to change the denominator \newline 
    Just multiply the numerator with the numerator and the denominator with the denominator \newline 
    Then you can reduce the final outcome if the directions of the problem requires you to \newline 
    \textbf{Example} \newline 
    \begin{center}
    $
    \frac{2}{24}*\frac{3}{5}=\frac{6}{120}
    $
    Reduced form $ \frac{1}{40}$
    \end{center}
    \item dividing fractions 
    You take the reciprocal of the dividend (flip the numerator and the denominator) and then multiply the two fractions \newline 
    \textbf{Example} \newline 
    \begin{center}
    $
    \frac{2}{6} \div \frac{4}{5} = 
    $
    \newline 
    \textbf{Flip the } \newline 
    $
    \frac{2}{6} \times \frac{5}{4} = \frac{10}{24}
    $
    \newline 
    reduced form of the answer: $ \frac{5}{12}
    $
    \end{center}
\end{itemize}

\section{Trigonometry}
\textbf{Basic Trig Functions}
\begin{itemize}
    \item sin(x)
    \item cos(x)
    \item tan(x)
    \item sec(x)
    \item csc(x)
    \item cot(x)
\end{itemize}
\textbf{Basic Trig Identities}\newline
\textbf{Quotient Identity}\newline 

    $
    tan(x) = \frac{sin(x)}{cos(x)} \newline
    cot(x) = \frac{cos(x)}{sin(x)} \newline 
    $

\textbf{Reciprocal Identity}\newline
\begin{itemize}
    \item $ csc(x) = \frac{1}{sin(x)}$
    \item $sec(x) = \frac{1}{cos(x)}$
    \item $cot(x) = \frac{1}{tan(x)}$
    \item $sin(x) = \frac{1}{csc(x)}$
    \item $cos(x) = \frac{1}{sec(x)}$
    \item $tan(x) = \frac{1}{cot(x)}$
    
\end{itemize}

\textbf{Pythagorean Identity}\newline 
\begin{itemize}
    \item $ sin^{2}(x)+cos^{2}(x) =1$
    \item $ sin^{2}(x) = 1-cos^{2}(x) $
    \item $ cos^{2}(x) = 1 - sin^{2}(x)$
    \item $tan^{2}(x) + 1 =  sec^{2}(x)$
    \item $sec^{2}(x)-1 = tan^{2}(x)$
\end{itemize}
\textbf{Double Angle Trig}
\begin{itemize}
    \item $ sin(2\theta) = 2sin(\theta)cos(\theta)$
    \item $ cos(2\theta) = cos^{2}(\theta)-sin^{2}(\theta)$
    \item $cos(2\theta) = 2cos^{2}(\theta) -1 $
\end{itemize}
\end{document}
